\title{
    Sistema de prevencion de muerte de cuna\\
    \small{Ingenieria en Diseño Electronico y Sistemas Inteligentes}
}
\author{
    \IEEEauthorblockN{Juan Daniel Polanco Avalos}
     \IEEEauthorblockA{
         18310246
     }
    \and
    \IEEEauthorblockN{Christopher Adan Madrigal Renteria}
     \IEEEauthorblockA{
         18310172
     }
    \and
    \IEEEauthorblockN{Daniel Rodriguez Contreras}
     \IEEEauthorblockA{
         18310282
     }
}
\maketitle

%
% Descripción del reporte
%
\begin{abstract}
    % Presentar una introducción y justficación a los propositos del proyecto de titulación 
    % \emph{``Prevención de muerte por asfixia accidental''}, enfocado a reducir el riesgo de muerte por asfixia 
    % monitoreando al lactante durante el periodo de siesta.

    Hoy en día el porcentaje de bebés afectados por el Síndrome de muerte súbita del lactante (SMSL) cada vez es más alto,
    con el presente proyecto se busca reducir este porcentaje con un sistema que monitoree las variables vitales que previenen el SMSL como lo son la posición, temperatura y respiracion del bebé,
    donde si hay alguna situación de peligro está sea notificada a los padres mediante una aplicación móvil.


\end{abstract}
\begin{IEEEkeywords}
    Muerte súbita del lactante, muerte infantil, posición del sueño, monitoreo de lactantes, prevención de SMSL.
\end{IEEEkeywords}

\tableofcontents
