\title{
    Sistema de prevencion de muerte de cuna\\
    \small{Ingenieria en Diseño Electronico y Sistemas Inteligentes}
}
\author{
    \IEEEauthorblockN{Juan Daniel Polanco Avalos}
    % \IEEEauthorblockA{
    %     \textit{Ingenieria en Diseño Electronico y Sistemas Inteligentes} \\
    %     Guadalajara, Jalisco \\
    %     jdanypa@gmail.com
    % }
    \and
    \IEEEauthorblockN{Christopher Adan Madrigal Renteria}
    % \IEEEauthorblockA{
    %     \textit{Ingenieria en Diseño Electronico y Sistemas Inteligentes} \\
    %     Guadalajara, Jalisco \\
    %     jdanypa@gmail.com
    % }
    \and
    \IEEEauthorblockN{Daniel Rodriguez Contreras}
    % \IEEEauthorblockA{
    %     \textit{Ingenieria en Diseño Electronico y Sistemas Inteligentes} \\
    %     Guadalajara, Jalisco \\
    %     jdanypa@gmail.com
    % }
}
\maketitle

%
% Descripción del reporte
%
\begin{abstract}
   % Presentar una introducción y justficación a los propositos del proyecto de titulación 
   % \emph{``Prevención de muerte por asfixia accidental''}, enfocado a reducir el riesgo de muerte por asfixia 
   % monitoreando al lactante durante el periodo de siesta.

   Presentar el Sistema de prevención de Síndrome de muerte súbita de lactantes, 
   enfocado a tratar de prevenirlo mediante un constante monitoreo durante el periodo de siesta.  
   En el presente decumento se mostrara las intenciones y objetivos del mismo, el desarrollo teorico
   y practico, su construccion y los resultados del desarrollo del mismo.


\end{abstract}
\begin{IEEEkeywords}
    Muerte súbita del lactante, muerte infantil, posición del sueño, monitoreo de lactantes, prevención de SIDS.
\end{IEEEkeywords}

\tableofcontents
