
\begin{thebibliography}{00}
    \bibitem{b1} Munkel Ramírez, Laura, Durón González, Rodrigo, Bolaños Morera, Pamela. (2018). Síndrome de muerte súbita del lactante. Medicina Legal de Costa Rica, 35(1), 65-74. Recuperado 28 de febrero de 2022, de \url{http://www.scielo.sa.cr/scielo.php?script=sci_arttext&pid=S1409-00152018000100065&lng=en&tlng=es}.
    \bibitem{b2} Kinney, H. C., Thach, B. T. (2009). The sudden infant death syndrome. The New England journal of medicine, 361(8), 795-805. \url{https://doi.org/10.1056/NEJMra0803836}
    \bibitem{b3} García, V. (2018, 2 marzo). Configurar el MPU6050. – Electrónica Práctica Aplicada. Electrónica Práctica Aplicada. Recuperado 22 de marzo de 2022, de \url{https://www.diarioelectronicohoy.com/blog/configurar-el-mpu6050}
    \bibitem{b4} Sensor de temperatura infrarrojo MLX90614. (s. f.). Naylamp Mechatronics - Perú. Recuperado 22 de marzo de 2022, de \url{https://naylampmechatronics.com/sensores-temperatura-y-humedad/330-sensor-de-temperatura-mlx90614.html}
    \bibitem{b5} Llamas, L. (2018, 4 abril). Arduino y el termómetro infrarrojo a distancia MLX90614. Luis Llamas. Recuperado 18 de abril de 2022, de \url{https://www.luisllamas.es/arduino-y-el-termometro-infrarrojo-a-distancia-mlx90614/}
    \bibitem{b6} Carmenate, J. G. (2022, 7 marzo). ESP32 Wifi + Bluetooth en un solo lugar. Programar fácil con Arduino. Recuperado 9 de mayo de 2022, de \url{https://programarfacil.com/esp8266/esp32/}
    \bibitem{b7} UNIT Electronics. (2022, 21 junio). Sensor de Temperatura Infrarrojo GY-906 MLX90614. Recuperado 30 de mayo de 2022, de url{https://uelectronics.com/producto/sensor-de-temperatura-infrarrojo-gy-906-mlx90614/}
\end{thebibliography}

\printglossary[type=\acronymtype]