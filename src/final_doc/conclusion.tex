\section{Conclusión}
En primer término, se arribó a la conclusión de que, en algunos de los casos, el sindrome de muerte subita del lactanate se puede prevenir.
Se logró elaborar un dispositivo capaz de detectar la posición del lactante al igual que la temperatura y su ritmo de respiración el cual apoya a los padres al monitoreo constante y alerta para la prevención de la muerte de cuna.

El prototipo presentado se diseño de la manera mas optima posible reciclando algunos componentes ya que hoy en dia habia un desabasto de componentes al cual nos tuvimos que ajustar como equipo al momento de desarrollar el dispositivo.

La experiencia captada, se cuestiona parcial o totalmente a lo aprendido dentro de nustros años cursados en el Centro de Enseñanza Tecnico Industrial,
ya que prototipo paso desde una fase de prueba con electronica base hasta el diseño completo de esquematicos y pcbs tanto mas complejas como el desarrollo de un propio sistema operativo dentro de nuestro microcontrolador.

Despues de la pruebas pertinentes y la implementación del prototipo se llego a la conclusión de que hay cosas que podrian mejorar en nuestro diseño que apoyaria de una mejor manera al monitoreo del lactante, pero el dispositivo es funcional para alertar a los padres sobre algun riesgo.