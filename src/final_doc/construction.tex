\section{Construcción}

    \subsection{Modulo de carga y alimentación}

    Dentro del sistema es critico el constante monitoreo de las variables de orientación, 
    temperatura y respiración del bebé, para ello se utilizó una pila Li-Po para mantener 
    alimentado el sistema. Al no poderse utilizar unicamente la batería de forma directa, se
    diseño el siguiente modulo de carga y alimentación, el cual esta construido por 3 partes:

    \begin{figure}[htp!]
        \centering
        \includegraphics[width=\columnwidth]{charging_and_power_supply_module.png}
        \caption{Partes que conforman el modulo de carga y alimentación}
        \label{fig: charging_and_power_module}
    \end{figure}
    \FloatBarrier

    \subsection{Diseño de PCB}

    \subsubsection{Esquematicos}   

        \begin{figure}[htp!]
            \centering
            \includegraphics[width=\columnwidth]{battery_module_schematic.png}
            \caption{Esquematicos del modulo de carga y alimentación}
            \label{fig: battery_module_schematic}
        \end{figure}
        \FloatBarrier

        \begin{figure}[htp!]
            \centering
            \includegraphics[width=\columnwidth]{esp32_schematic.png}
            \caption{Esquematico del ESP32-WROOM}
            \label{fig: esp32_schematic}
        \end{figure}
        \FloatBarrier

        \begin{figure}[htp!]
            \centering
            \includegraphics[width=\columnwidth]{ft232rl_schematic.png}
            \caption{Esquematico del FT232RL}
            \label{fig: ft232rl_schematic}
        \end{figure}
        \FloatBarrier

        \begin{figure}[htp!]
            \centering
            \includegraphics[width=\columnwidth]{sensors_schematics.png}
            \caption{Esquematicos de los sensores}
            \label{fig: sensors_schematics}
        \end{figure}
        \FloatBarrier

        

    \subsubsection{Layout}
        Al diseñar los diagramas esquematicos y realizando las pruebas pertinentes se decidio realizar 
        una unica pcb la cual seria manera pequeña y que no representara problemas o incomodidad por su 
        tamaño, forma y carcasa al bebé.

        A continuación se muestra el layout final de la primera versión:

        \begin{figure}[htp!]
            \centering
            \includegraphics[width=\columnwidth]{layout.jpeg}
            \caption{Layout del PCB}
            \label{fig: layout}
        \end{figure}
        \FloatBarrier

    \subsubsection{PCB 3D}
        \begin{figure}[htp!]
            \centering
            \includegraphics[width=\columnwidth]{isometric_view_3D.jpg}
            \caption{Vista isometrica del PCB y la batería}
            \label{fig: isometric_3d}
        \end{figure}
        \FloatBarrier
        \begin{figure}[htp!]
            \centering
            \includegraphics[width=\columnwidth]{top_view_3D_1.jpg}
            \caption{Vista superior del PCB}
            \label{fig: top_3d}
        \end{figure}
        \FloatBarrier

        En las figuras \ref{fig: isometric_3d} y \ref{fig: top_3d} se muestra el modelo 3D, este fue hecho
        para poder corraborar de que tamaño quedaría el dispositivo final contando la PCB y la batería.

        Tanto el modelo 3D como el layout de la PCB fue diseñado en el software Altium.
    \subsection{Diseño de Carcasa}
        \begin{figure}[htp!]
            \centering
                \includegraphics[width=\columnwidth]{system_shell_v2.png}
                \caption{Vista isometrica de la carcasa}
                \label{fig: isometric_view_shell}
        \end{figure}
        \FloatBarrier

        \begin{figure}[htp!]
            \centering
                \includegraphics[width=\columnwidth]{front_view_system_shell_v2.png}
                \caption{Vista de la parte frontal de la carcasa, donde esta el puerto USB}
                \label{fig: front_view_shell}
        \end{figure}
        \FloatBarrier

        \begin{figure}[htp!]
            \centering
                \includegraphics[width=\columnwidth]{top_view_syste_shell_v2.png}
                \caption{Vista superior de la carcasa}
                \label{fig: top_view_shell}
        \end{figure}
        \FloatBarrier

        \begin{figure}[htp!]
            \centering
                \includegraphics[width=\columnwidth]{system_shell v2_w_pcb.png}
                \caption{Vista isometrica de la carcasa con la PCB y bateria dentro}
                \label{isometric_view_pcb_shell}
        \end{figure}
        \FloatBarrier

        En las figuras \ref{fig: isometric_view_shell}, \ref{fig: front_view_shell},
        \ref{fig: top_view_shell} y \ref{isometric_view_pcb_shell}, se puede ver el diseño propuesto
        para la carcasa del sistema, donde esta tendra las siguientes aberturas:

        \begin{itemize}
            \item Switch encendido/apagado
            \item Puerto de carga Micro-USB tipo-B
            \item Salida para el sonido del buzzer
            \item Led infrarrojo del sensor de temperatura
        \end{itemize}

